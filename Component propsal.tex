\documentclass{scrartcl}

\usepackage[hidelinks]{hyperref}

\title{COMP220 - Portfolio of game engine components - graphics}

\author{1505536}

\begin{document}

\maketitle

\subsection{Title and concept }
The title of this demo is Woodland and within this concept the player will start in a forest by a camp site and follow a process of exploring the terrain. The player is able to interact with the surrounding environment in various ways including exploring, through walking and jumping across the differing terrain and handling objects. These objects will further interaction with the player and can be used to throw with the example of branches, rocks and wood. One specific example will be the use of wood on a fire at the camp site.
\\
\\
\subsection{Aesthetic}
The intended aesthetic will be a very bright soft colours of the forest encouraging the player to explore the environment around them. The colours described will predominantly be soft summer colours including light greens, yellows and light blue for the sky. The plants will be in bloom so a range of colours will be added here to make the atmosphere more visually pleasing and allowing the player to explore all elements of this terrain. The sun will be at mid-day permanently as it will provide the game with the most light, voiding the concept of time. In terms of the terrain there will be a wide range of gradients of hills and will show areas of thick tall trees and shrubbery in areas alongside open areas with water elements. This provides the player a wide range to explore throughout the demo and to ensure a realistic interest with this theme. 
\\
\\
\\
\subsection{Components}
Collision detection will be used in the Woodland demo which will allow the player to jump and walk into objects giving the forest a realistic feel when the player explores it. This realistic approach will encourage the player to explore as if they were walking through the terrain in person and will block the player from walking through objects. This idea will also mean that bushes and branches will move when touched which adds to the interaction with the player. 
\\
The forest will also have particle effects on it which will include effects like smoke from an old camp fire and particles in the air. This will add more depth to the world making it more visually appealing which encourages the player to explore more. It will also provide an effect which will make the world look magical and to create the sensation of a fantasy.



\subsection{Scope}
Overall the scope for this demo is a forest terrain that can be explored by the player and encouraging interaction through the use and handling of objects and collision detection. This concept is relatively simple allowing the focus to be on the exploration of the terrain instead of complicating the user with multiple mechanics. This scope will be appropriate for the time frame of this project due to the world being procedurally generated terrain.
\end{document}