\documentclass{scrartcl}

\usepackage[hidelinks]{hyperref}

\title{COMP220 - Portfolio of game engine components - graphics}

\author{1505536}

\begin{document}

\maketitle

\section{Title and concept }
The title for my demo will be rubble. The concept will be a cave which has had the entrance sealed off by rubble which has fallen. The cave will be quite large with long tunnels and open rooms. The player will be able to move throughout the cave in order to inspect all the tunnels and rooms.
\\
\\
\section{Aesthetic}
The cave will have long and windy tunnels which interconnect and lead the player to big open rooms. The cave will be spacious with high ceiling that have rocks hanging from the ceiling as well as rocks and boulders around the cave to give the cave depth. The caves lighting will be dim throughout in order to create quite a dark mysterious. This will add an air of mystery to the cave before the player explores. It will have a similar feel to the caves in skyrim. Quite dominating like anything could be in the cave and you have to be careful when exploring it.
\\
\\
\\
\section{Components}
\subsection{procedural generation}
For my cave it is going to be procedurally generated. This means that the cave will change each time which means it will be a new environment to explore each time. The caves generation will follow rules which mean that it will always have rooms and tunnels and they will all link together. They will also loop so that the player can go round in circles giving the cave the feel of a whole tunnel system. This brings a good element to the exploring part of the game as the cave will change each time and they can be anything within the rules for generating the cave.
\subsection{texturing}
The cave in the demos aesthetic will be made with various textures which will change depending on the surface. I will have the textures load in each time the cave is procedurally generated. They will make the cave look more realistic and will make it  more interesting encouraging the player to explore it. This adds to the aesthetic of my game in order to make it more visually appealing.


\section{Scope}
I think the scope for this project is appropriate. This is because I have got two key components which I feel I can achieve and then beyond this I would like to have a stretch goal to add collision detection. However this will just be a stretch goal if my main components are achieved. 
\end{document}